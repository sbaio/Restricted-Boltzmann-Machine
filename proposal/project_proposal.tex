\documentclass{article}
\usepackage[utf8]{inputenc}
%\usepackage[latin]{inputenc}
\usepackage[english]{babel}
\usepackage[T1]{fontenc}
\usepackage{tgtermes}
\usepackage{float, caption}
\usepackage[dvips]{color}
\usepackage{amsfonts}
\usepackage{amsmath}
%\usepackage{bbold}
%\usepackage{multirow}
\usepackage{amsthm,amssymb}
\renewcommand{\qedsymbol}{$\blacksquare$}
\newcommand\independent{\protect\mathpalette{\protect\independenT}{\perp}}
\def\independenT#1#2{\mathrel{\rlap{$#1#2$}\mkern2mu{#1#2}}}


\usepackage{xspace}
\usepackage{array}
\usepackage{graphicx}
\usepackage{latexsym}
\usepackage{mathtools}
\usepackage{xcolor}
\newcommand\myworries[1]{\textcolor{red}{#1}}
\newcommand{\todo}[1]{\textcolor{red}{TODO: #1}\PackageWarning{TODO:}{#1!}}
\usepackage{listings}
\lstset{language=Matlab}
\lstset{breaklines}
\lstset{extendedchars=false}
  \author{Xi SHEN, Othman SBAI, Chaïmaa KADAOUI}
  \title{Restricted Boltzmann Machine - Project proposal}
  \date{Dec, 2016}
\begin{document}
\maketitle
\section{Introduction}
Restricted Boltzmann Machines are generative models based on hidden variables to model a data distribution.

\section{Plan of work}
This sentence was generated from terminal, with gitignore 1
\todo{Add Content}

\section{Dataset}

\todo{Add Content}

\section{Result}

\todo{Add Content}


\todo{Detail more references}
\begin{thebibliography}{9}
	\setlength{\parskip}{0pt} 
	
	\bibitem{praticalGuide} Hinton G. \textit{ A practical guide to training restricted Boltzmann machines[J]. Momentum, 2010, 9(1): 926.}
	
	\bibitem{these} Fischer A. \textit{Training restricted boltzmann machines[J]. KI-Künstliche Intelligenz, 2015, 29(4): 441-444.}
	
\end{thebibliography}
\end{document}